\documentclass[10pt,a4paper]{report}

\usepackage[francais]{babel}
\usepackage[T1]{fontenc}
\usepackage[utf8]{inputenc}

\usepackage{listings}
\usepackage{xcolor}
\usepackage{geometry}
\usepackage{fancyhdr}
\usepackage{graphicx}
\usepackage{hyperref}
\usepackage{lipsum}
\usepackage{setspace}

\hypersetup{
    colorlinks,
    citecolor=black,
    filecolor=black,
    linkcolor=black,
    urlcolor=black
}


%%configuration de listings
\lstset{
language=Prolog,
basicstyle=\ttfamily\small, %
identifierstyle=\color{red}, %
keywordstyle=\color{blue}, %
numberstyle=\color{green},
stringstyle=\color{black!60}, %
commentstyle=\it\color{green}, %
columns=flexible, %
tabsize=2, %
extendedchars=true, %
showspaces=false, %
showstringspaces=false, %
numbers=left, %
numberstyle=\tiny, %
breaklines=true, %
breakautoindent=true, %
captionpos=b,
backgroundcolor=\color{Zgris},
captionpos = b 
}

\definecolor{Zgris}{rgb}{0.87,0.85,0.85}
\newcommand{\HRule}{\rule{\linewidth}{1mm}}
\newsavebox{\BBbox}
\newenvironment{DDbox}[1]{
\begin{lrbox}{\BBbox}\begin{minipage}{\linewidth}}
{\end{minipage}\end{lrbox}\noindent\colorbox{Zgris}{\usebox{\BBbox}} \\
[.5cm]}

% Redefine the plain page style

%\headsep = 25pt

\fancypagestyle{plain}{%
  \fancyhf{}
%header
  \fancyhead[L]{Rapport de projet}
  \fancyhead[R]{\ifnum\value{part}>0 \partname\ \thepart \fi}
  \renewcommand{\headrulewidth}{0.4pt}% Line at the header visible
%footer
  \fancyfoot[L]{Baptiste Lesquoy et Nicolas Weissenbach}
  \fancyfoot[C]{\thepage}
  \fancyfoot[R]{\today}%
  \renewcommand{\footrulewidth}{0.4pt}% Line at the footer visible
}


 \geometry{
 a4paper,
 total={210mm,297mm},
 left=20mm,
 right=20mm,
 top=20mm,
 bottom=20mm,
 }




\begin{document}

%%
%%	TITLE PAGE
%%
  \begin{titlepage}
    \begin{center}
      % Upper part of the page. The '~' is needed because \\
      % only works if a paragraph has started.
      ~\\[4cm]
      \textsc{\LARGE Université de Lorraine}\\[1.5cm]
      \textsc{\Large Faculté de Science et Technologie}\\[0.5cm]
      \textsc{\Large Master 1 Informatique}\\[0.5cm]
      % Title
      \HRule \\[0.4cm]
      { \huge \bfseries Projet de LMC \\[0.4cm] }
      \HRule \\[1cm]
     \textsc{\LARGE Algorithme d'unification \\ ~ \\ Martelli Montanari }\\[1cm]
      \begin{tabular*}{\textwidth}{@{}l@{\extracolsep{\fill}}r@{}}
	\emph{Auteurs:}\\
	Baptiste \textsc{Lesquoy} \\
	Nicolas \textsc{Weissenbach}\\
      \end{tabular*}
      \vfill
      % Bottom of the page
      {\large \today}
      %\LaTex
    \end{center}
  \end{titlepage}

\tableofcontents
\chapter*{Question 1}

\section*{Les premiers pas vers l'unification}
Dans un premier temps, nous avons décidé d'écrire tous les prédicats de règle (qui permettent de savoir quelle règle utiliser) et les prédicats de réduction. Commencer par ces prédicats aura permis d'avoir une base de travaille et d'emprunter un processus de développement itératif.

\paragraph{Exemple d'un prédicat $regle$} ~\\

\begin{lstlisting}[caption ={Choix de la règle $simplify$}]
regle(E, simplify):-
	split(E, L, R),
	not(var(R)),
	var(L),
	not(compound(R))
.
\end{lstlisting}
Ici nous avons le prédicat regle pour l'unification avec $simplify$, le but de ce prédicat est de renvoyer vrai si simplify est applicable à une equation $E$. Le prédicat est faux si la règle n'est pas applicable. \\
Toutes les règles on été instanciées pour chacune des unifications possible.

\paragraph{Exemple d'un prédicat $reduit$} ~\\

\begin{lstlisting}[caption ={Application de la réduction $expand$}]
reduit(expand, E, P, Q) :-
	split(E, X, T),
	X = T,
	delete_elem(E, P, Q)
.
\end{lstlisting}
Le prédicat $reduit$ permet d'appliquer la réduction approprié à l'équation $E$. Ici le prédicat $reduit$ applique la transformation $expand$ à l'équation $E$ et crée $Q$ le nouvel ensemble de travail. 

\section*{Unification}
A la suite de l'implémentation de ces prédicats, nous avons mis en place le système de prédicat qui permet résoudre l'unificateur le plus général. Pour cela, nous avons implémenté le prédicat $unifie$. C'est ce prédicat qui fait le lien entre les prédicats de choix de transformation ($regle$) et ceux d'application de cette transformation.

\begin{lstlisting}[caption ={Prédicat d'unification d'un système d'équation}]
unifie([E|P]) :-
write("system: "),print([E|P]),nl,
regle(E, R),
write(R), write(": "), write(E), nl,
reduit(R, E, P, Q),
unifie(Q)
.
\end{lstlisting}


\chapter*{Question 2}
La rapidité d'exécution de l'algorithme dépend du choix plus ou moins judicieux des règles à exécuter en premier sur le système d'équation. C'est pourquoi il faut mettre en place un système qui prenne en compte le l'ordre dans lequel les équations (ou les règles) sont choisies.
Nous avons donc décidé d'implémenter quatre méthodes de choix :
\begin{itemize}
\item Choix du premier : la première équation de l'ensemble est résolue
\item Choix pondéré : les équations sont choisies en fonction des règles que l'on peut appliquer dessus. Les règles sont classées dans un ordre préétablit qui est plus judicieux dans certains cas . Pour notre exemple les règles $clash$ et $check$ sont prioritaire, ce qui permet d'arrêter l'exécution de l'algorithme au plus tôt si jamais le système d'équation est insolvable.
\item Choix aléatoire : les équations sont prise de façon aléatoire, on y applique la règle qui est possible
\item Choix du dernier : la dernière équation du système est choisie
\end{itemize}

Grâce à ces quatre possibilités, nous pouvons voir la différence d'exécution entre ces différentes méthodes de choix d'équation :
\begin{lstlisting}[caption ={Exemple d'exécution avec le choix de la première équation }]
3 ?- trace_unif([f(X, Y, h(C, V)) ?= f(g(Z), h(a), h(e, r)), Z ?= f(Y), f(g) ?= f(a, b)], premier).
system: [f(_G1995,_G1996,h(_G1992,_G1993))?=f(g(_G1999),h(a),h(e,r)),_G1999?=f(_G1996),f(g)?=f(a,b)]
decompose: f(_G1995,_G1996,h(_G1992,_G1993))?=f(g(_G1999),h(a),h(e,r))
system: [_G1995?=g(_G1999),_G1996?=h(a),h(_G1992,_G1993)?=h(e,r),_G1999?=f(_G1996),f(g)?=f(a,b)]
expand: _G1995?=g(_G1999)
system: [_G1996?=h(a),h(_G1992,_G1993)?=h(e,r),_G1999?=f(_G1996),f(g)?=f(a,b)]
expand: _G1996?=h(a)
system: [h(_G1992,_G1993)?=h(e,r),_G1999?=f(h(a)),f(g)?=f(a,b)]
decompose: h(_G1992,_G1993)?=h(e,r)
system: [_G1992?=e,_G1993?=r,_G1999?=f(h(a)),f(g)?=f(a,b)]
simplify: _G1992?=e
system: [_G1993?=r,_G1999?=f(h(a)),f(g)?=f(a,b)]
simplify: _G1993?=r
system: [_G1999?=f(h(a)),f(g)?=f(a,b)]
expand: _G1999?=f(h(a))
system: [f(g)?=f(a,b)]
clash: f(g)?=f(a,b)
No

\end{lstlisting}

\begin{lstlisting}[caption ={Exemple d'exécution avec le choix pondéré}]
4 ?- trace_unif([f(X, Y, h(C, V)) ?= f(g(Z), h(a), h(e, r)), Z ?= f(Y), f(g) ?= f(a, b)], pondere).
system: [f(_G1995,_G1996,h(_G1992,_G1993))?=f(g(_G1999),h(a),h(e,r)),_G1999?=f(_G1996),f(g)?=f(a,b)]
clash: f(g)?=f(a,b)
No
\end{lstlisting}

\begin{lstlisting}[caption ={Exemple d'exécution avec le choix aléatoire}]
5 ?- trace_unif([f(X, Y, h(C, V)) ?= f(g(Z), h(a), h(e, r)), Z ?= f(Y), f(g) ?= f(a, b)], aleatoire).
system: [f(_G2007,_G2008,h(_G2004,_G2005))?=f(g(_G2011),h(a),h(e,r)),_G2011?=f(_G2008),f(g)?=f(a,b)]
expand: _G2011?=f(_G2008)
system: [f(_G2007,_G2008,h(_G2004,_G2005))?=f(g(f(_G2008)),h(a),h(e,r)),f(g)?=f(a,b)]
clash: f(g)?=f(a,b)
No

\end{lstlisting}

On constate donc que sur ce cas particulier, le choix pondéré a bien fonctionné, c'est à dire que l'algorithme a tout de suite repéré que le système était insolvable et a stoppé l'exécution, le choix aléatoire s'en sort bien lui aussi mais ça aurait très bien pu ne pas être le cas, le choix de la première équation est le pire des cas puisqu'on ne tombe sur le problème qu'une fois toutes les autres équations résolues.

\chapter*{Question 3}

Pour passer du fonctionnement de la question 2 à celui de la question 3. C'est-à-dire créer les prédicats $unif(P,S)$ et $trace_unif(P,S)$ permettant respectivement d'exécuter unifie sans et avec les messages dans la console. Il a suffit de remplacer tous les appels aux fonctions "write" et "print" par un appel à la fonction "echo", les nouveaux prédicats font tout deux appels à unifie, mais ils précèdent cet appel par clr\_echo ou set\_echo, servant respectivement à désactiver ou activer le flag echo\_on. Ainsi quand le prédicat unifie appellera "echo" ce dernier affichera ou n'affichera pas le message en fonction de l'état du flag mais continuera son exécution quoi qu'il arrive.

\begin{lstlisting}[caption ={Exemple d'execution avec un niveau de debug}]
?- trace_unif([a ?= U, f(X, Y) ?= f(g(Z ),h(a )), Z ?= f(Y )], premier).
system: [a?=_G192,f(_G197,_G198)?=f(g(_G200),h(a )),_G200?=f(_G198)]
orient: a?=_G192
system: [_G192?=a,f(_G197,_G198)?=f(g(_G200),h(a )),_G200?=f(_G198)]
simplify: _G192?=a
system: [f(_G197,_G198)?=f(g(_G200),h(a )),_G200?=f(_G198)]
decompose: f(_G197,_G198)?=f(g(_G200),h(a ))
system: [_G197?=g(_G200),_G198?=h(a),_G200?=f(_G198)]
expand: _G197?=g(_G200)
system: [_G198?=h(a ),_G200?=f(_G198)]
expand: _G198?=h(a )
system: [_G200?=f(h(a ))]
expand: _G200?=f(h(a ))
Yes
U = a,
X = g(f(h(a ))),
Y = h(a ),
Z = f(h(a )).
\end{lstlisting}

\begin{lstlisting}[caption ={Exemple d'exécution }]
unif([a ?= U, f(X, Y) ?= f(g(Z),h(a)), Z ?= f(Y)], premier).
U = a,
X = g(f(h(a))),
Y = h(a),
Z = f(h(a)).
\end{lstlisting}
\appendix
\chapter{Sources}
Les sources de ce projet sont divisées en 2 fichiers. Le fichier principal est le fichier $projet.pl$, il contient le programme demandé. Le second fichier est le fichier $test.pl$ qui contient les tests unitaires que nous avons créés pour s'assurer que notre programme faisait bien le travail voulu, notamment les prédicats $regle$ et $reduit$. Les tests se lance avec le prédicat $tests.$. Les sources sont aussi consultables en ligne à l'adresse \url{www.github.com/lesquoyb/martelli_montanari} .

\section{projet}
\begin{lstlisting}[caption ={le fichier projet.pl}]
:- op(20,xfy,?=).

% Predicats d'affichage fournis

% set_echo: ce predicat active l'affichage par le predicat echo
set_echo :- assert(echo_on).

% clr_echo: ce predicat inhibe l'affichage par le predicat echo
clr_echo :- retractall(echo_on).

% echo(T): si le flag echo_on est positionne, echo(T) affiche le terme T
%          sinon, echo(T) reussit simplement en ne faisant rien.

echo(T) :- echo_on, !, write(T).
echo(_). 

%select_strat choisi une equation a l'aide de predicat correspondant a la strategie passee en parametre 
select_strat(premier, P, E, R):-
	choix_premier(P, _, E, R)
.

select_strat(pondere, P, E, R):-
	choix_pondere(P, _, E, R)
.

select_strat(aleatoire, P, E, R):-
	choix_aleatoire(P, _, E, R)
.
select_strat(dernier, P, E, R):-
	choix_dernier(P, _, E, R)
.

%le coeur du programme, essaye d'unifier P en utilisant la strategie Strat, si on ne passe pas de strategie il choisira la strategie "premiere equation"
unifie(P):- unifie(P, premier).
unifie([], _):- true, !.
unifie(bottom, _):- false, !. %Si on a bottom => c'est un echec
unifie(P, Strat):-
	echo("system: "),echo(P),echo("\n"),
	select_strat(Strat, P, E, R),
	echo(R),echo(": "),echo(E),echo("\n"),
	reduit(R, E, P, Q),
	unifie(Q, Strat), !
.
%appelle unifie apres avoir desactive les affichages
unif(P, S):-
	clr_echo,
	unifie(P, S)
.
%appelle unifie apres avoir active les affichages, affiche "Yes" si on peut unifier "No" sinon (il n'y a donc pas d'echec de la procedure.
trace_unif(P,S):-
	set_echo,
	(
		unifie(P, S), echo("Yes"),!
	;	
		echo("No")
	)
.
%sous fonction du predicat de selection d'equation avec choix pondere
select_rule([], _, _, _):- false, !.%on a parcouru la liste des regle sans en trouver une qui fonctionne
select_rule( [Next |  MasterList], [], Pbase, P, R, E):-%on a vide le sous groupe de regle qu'on etait en train de traiter, on passe au suivant
	select_rule(MasterList, Next, Pbase, P, R, E)
.
select_rule(MasterList, [ _ | ListRules], Pbase, [], R, E):-%on a parcourus toutes les equations sans pouvoir appliquer la regle voulue, on passe a la regle suivante, et on reinitialise les equations
	select_rule(MasterList, ListRules, Pbase, Pbase, R, E)
.
select_rule( MasterList,[FirstRule | ListRules], Pbase, [Ep|P], R, E):-%on essaye d'appliquer la regle FirstRule a Ep, si on echoue, on recommence avec l'equation suivante
	(
		regle(Ep, FirstRule),
		R = FirstRule,
		E = Ep,
		!
	;
		select_rule(MasterList, [FirstRule | ListRules], Pbase, P, R, E),!
	)
.
%remplie la liste pondere de regle, il s'agit d'une liste de liste. Chaque sous liste represente un groupe de regles de meme importance (oui on aurait pu faire une simple liste, mais c'est surement plus evolutif comme ca )
liste_pondere([ [clash, check],
				[rename, simplify],
				[orient, decompose],
				[expand]
			  ]):- 
	true 
.
%choix de la premiere equation dans P
choix_premier( [E|_], _, E, R):- 
	regle(E,R),!	
.
%choix de la premiere equation satisfaisant la regle de plus haute importance
choix_pondere(P, _, E, R):-
	liste_pondere( [FirstRules | List] ),
	select_rule(List,FirstRules, P, P, R, E)
.
%choix d'une equation aleatoire
choix_aleatoire(P, _, E, R):-
	random_member(E, P),
	regle(E, R),
	!
.
%choix de la derniere equation dans P
choix_dernier(P, _, E, R):-
	reverse(P, [E|_]),
	regle(E, R),
	!
.
%regle teste si on peut appliquer la regle R (deuxieme parametre) a l'equation E (premier parametre)
regle(E, decompose):-
	not(atom(E)),
	split(E, X, Y),
	compound(X),
	compound(Y),
	compound_name_arity(X, N, A),
	compound_name_arity(Y, N, A)
.
regle(E, simplify):-
	split(E, L, R),
	var(L),
	not(var(R)),
	not(compound(R))
.
regle(E, rename):-
	split(E, L, R),
	var(R),
	var(L)
.
regle(E, expand):-
	split(E, X, Y),
	var(X),
	compound(Y),
	not(occur_check(X, Y))
.
regle(E, clash):-
	split(E, L, R),
	compound(L),
	compound(R),
	compound_name_arity(L, Nl, Al),
	compound_name_arity(R, Nr, Ar),
	not( (Nl == Nr, Al == Ar) )
.
regle(E, check):-
	split(E, X, Y),
	not(X == Y),
	var(X),
	occur_check(X, Y)
.
regle(E, orient):-
	split(E, L, R),
	var(R),
	not(var(L))
.

%Retourne L et R les arguments 1 et 2 de E (utilise pour couper une expression de type "X ?= Y" en deux)
split(E, L, R):-
	arg(1, E, L),
	arg(2, E, R)
.
%est vrai si V est une variable qui apparait dans T
occur_check(V, T) :-
	var(V),
	compound(T),
	term_variables(T, L),
	occur_check_list(V, L)
.
%sous fonction qui verifie parametre par parametre de T si V apparait
occur_check_list(_, []):-%on a parcouru tous les parametres sans rien trouver
	not(true)%parce que pourquoi avoir une constante false quand on peut faire not(true)
.
occur_check_list(V, [C|T]) :-%recursivite pour avancer dans la liste
	occur_check_list(V, T);
	V == C
.
%transforme f(x1,x2...,xn) ?= f(y1,y2,...,yn) en [x1 ?= y1, x2 ?= y2, ... , xn ?= yn] 
unif_list([], [],[]):- true.
unif_list([L|List1], [R|List2], [L ?= R| Rp]):-
	unif_list(List1, List2, Rp)
.
%reduit applique la regle R (premier argument) a l'equation E appartenant a P, et renvoie le nouvel ensemble Q
reduit(simplify, E, P, Q) :-
	split(E, X, T),
	X = T,
	delete_elem(E, P, Q)
.
reduit(rename, E, P, Q):-
	split(E, X, T),
	X = T,
	delete_elem(E, P, Q)
.
reduit(expand, E, P, Q) :-
	split(E, X, T),
	X = T,
	delete_elem(E, P, Q)
.
reduit(check, _, _, bottom):- false . 

reduit(orient, E, P, [ R ?= L | Tp ]) :-
	split(E, L, R),
	delete_elem(E, P, Tp)
.
reduit(decompose, E, P, S):-
	split(E, L, R),
	L =.. [_|ArgsL],
	R =.. [_|ArgsR],
	unif_list(ArgsL, ArgsR, Res),
	delete_elem(E, P, Pp),
	union(Res, Pp, S)
.
reduit(clash, _, _, bottom):- false .
%supprime l'element Elem de la list en deuxieme parametre et renvoie Set le nouvel ensemble
delete_elem(_, [], []) :- !.
delete_elem(Elem, [Elem|Set], Set):- ! .
delete_elem(Elem, [E|Set], [E|R]):-
	delete_elem(Elem, Set, R)
.
\end{lstlisting}
\section{tests}
\begin{lstlisting}[caption ={le fichier test.pl}]
:-
	op(20, xfy,[?=]),
	[projet]
.

writeOK() :- write(" : ok"), nl.

tests() :-
	writeln("Debut des tests : "), nl,
	writeln("==== Test : Reduit ===="), nl,
	test_reduit_decompose,
	test_reduit_rename,
	test_reduit_simplify,
	test_reduit_expand,
	test_reduit_orient,
	test_reduit_clash,
	test_reduit_check,
	writeln("Reduit : checked"),

	nl, writeln("==== Test : Regle ===="), nl,
	test_regle_simplify,
	test_regle_rename,
	test_regle_expand,
	test_regle_orient,
	test_regle_decompose,
	test_regle_clash,
	test_regle_check,
	writeln("Regle : checked"),
	/*
	On a supprime car le fonctionnnement de unifie a change au cours du TP, et ce n'est pas necessaire que ce soit teste car les fonctions qui constituent unifie sont testees independemment.
	nl,writeln("====== Test: Unifie ===="),nl,
	test_unifie(),
	*/
	write("Tous les tests sont passe avec succes")
.

test_unifie():-
	unifie([f(X, Y) ?= f(g(Z), h(a)), Z ?= f(Y)]),
	write("exemple prof juste: ok"),nl,nl,
	not(unifie([f(X, Y) ?= f(g(Z), h(a)), Z ?= f(X)])),
	write("exemple prof faux: ok"),nl,nl,
	not(unifie([a?=b])),
	not(unifie([a ?= X, X ?= b]))
.

test_reduit_decompose():-
	writeln("==== Reduit : Decompose ===="),

	write("f(a) ?= f(b), [a ?= b]"),
	reduit(decompose, f(a) ?= f(b), [f(a)?=f(b)], [a?=b]),
	writeOK,
	
	write("f(a) ?= f(a), [a ?= a]"), 
	reduit(decompose, f(a) ?= f(a), [], [a?=a]),
	writeOK,

	write("f(a,b,c) ?= f(d, e, f), [a ?= d, b ?= e, c ?= f]"), 
	reduit(decompose, f(a, b, c) ?= f(d, e, f), [], [a?=d, b ?= e, c ?= f ]),
	writeOK,


	write("f(X) ?= f(a), [X ?= a]"), 
	reduit(decompose, f(X) ?= f(a), [], [X?=a]),
	writeOK,


	write("f(a, X , b, Y) ?= f(d, e, f, g), [a ?= d, X ?= e, b ?= f, Y?= g]"), 
	reduit(decompose, f(a, X, b, Y) ?= f(d, e, f, g), [], [a?=d, X ?= e, b ?= f, Y ?= g]),
	writeOK,


write("f(g(X), W) ?= f(A, Q), [g(X) ?= A, W ?= Q]"),
	reduit(decompose, f(g(X), W) ?= f(A, Q), [f(g(X), W) ?= f(A, Q)], [g(X)?=A,W?=Q]),
	writeOK
.

test_reduit_rename() :-
	writeln("==== Reduit : Rename ===="),
	write("X?=Y, []"),
	reduit(rename, X2?=Y2, [X2?=Y2], []),
	X2 == Y2,
	writeOK,

	write("X?=Y,[X ?= a] => [Y ?= a]"),
	reduit(rename, X1?=Y1, [X1?=Y1, X1 ?= a], [Y1 ?= a]),
	X1 == Y1,
	writeOK

.

test_reduit_simplify() :-
	writeln("==== Reduit : Simplify ===="),
	
	write("X?=a , []"),
	reduit(simplify, X?=a, [X?=a], []),
	writeOK,
	write("X?=a,[Y ?= X] => [Y ?= a]"),
	reduit(simplify, X1?=a, [Y1?=X1, Y1 ?= X1], [Y1 ?= a]),
	writeOK
.

test_reduit_expand() :-
	writeln("==== Reduit : Expand ===="),

	write("X?=f(a), []"),
	reduit(expand, X2?=f(a), [X2?=f(a)], []),
	X2 == f(a),
	writeOK,

	write("X?=f(E), []"),
	reduit(expand, X3?=f(E2), [X3?=f(E2)], []),
	X3 == f(E2),
	writeOK
.

test_reduit_orient() :-
	writeln("==== Reduit : Orient ===="),

	write("f(W)?=X, [X?=f(W)]"),
	reduit(orient, f(W)?=X, [f(W)?=X], [X?=f(W)]),
	writeOK,

	write("f(a)?=X, [X?=f(a)]"),
	reduit(orient, f(a)?=X, [f(a)?=X], [X?=f(a)]),
	writeOK
.

test_reduit_clash() :-
	writeln("==== Reduit : clash ===="),
	
	write("clash"),
	not(reduit(clash, _, _, bottom)),
	writeOK
.

test_reduit_check() :-
	writeln("==== Reduit : check ===="),

	write("check"),
	not(reduit(check, _, _, bottom)),
	writeOK
.

test_regle_simplify() :- 
	writeln("==== Regle : simplify ===="),

	write("X ?= a"),
	regle(X?=a, simplify),
	writeOK,

	write("X ?= 1"),
	regle(X44 ?= 1, simplify),
	writeOK,


	write("a ?= b"),
	not(regle(a ?= b, simplify)),
	writeOK,

	write("X ?= f(a)"),
	not(regle(X1 ?= f(a), simplify)),
	writeOK,

	write("X ?= Y"),
	not(regle(X2 ?= Y, simplify)),
	writeOK,


	write("X ?= f(Y)"),
	not(regle(X3 ?= f(Y), simplify)),
	writeOK,

	write("f(a) ?= W"),
	not(regle(f(a) ?= W1, simplify)),
	writeOK,

	write("f(a) ?= f(W)"),
	not(regle(f(a) ?= f(W2), simplify)),
	writeOK
.

test_regle_rename() :-
	writeln("==== Regle : rename ===="),

	write("X ?= a"),
	not(regle(X1?=a, rename)),
	writeOK,

	write("X ?= f(a)"),
	not(regle(X2 ?= f(a), rename)),
	writeOK,

	write("X ?= Q"),
	regle(X3?=Q1, rename),
	writeOK,

	write("a ?= W"),
	not(regle(a ?= W1, rename)),
	writeOK,

	write("a ?= b"),
	not(regle(a ?= b, rename)),
	writeOK,

	write("a ?= f(b)"),
	not(regle( a ?= f(b), rename)),
	writeOK,


	write("f(a) ?= W"),
	not(regle(f(a) ?= W2, rename)),
	writeOK,

	write("f(a) ?= f(W)"),
	not(regle(f(a) ?= f(W3), rename)),
	writeOK
.

test_regle_expand() :-
	writeln("==== Regle : expand ===="),

	write("X ?= a"),
	not(regle(X1 ?= a, expand)),
	writeOK,

	write("X ?= f(a)"),
	regle(X2 ?= f(a), expand),
	writeOK,

	write("X ?= f(Q)"),
	regle(X3 ?= f(Q1), expand),
	writeOK,


	write("X ?= f(X)"),
	not(regle(X10 ?= f(X10), expand)),
	writeOK,


	write("X ?= f(a, b, X)"),
	not(regle(X12 ?= f(a, b, X12), expand)),
	writeOK,

	write("X ?= f(a, b, c, g(X))"),
	not(regle(X11 ?= f(a, b, c,g(X11) ), expand)),
	writeOK,

	write("X ?= Q"),
	not(regle( X4 ?=Q2, expand)),
	writeOK,

	write("f(a) ?= W"),
	not(regle(f(a) ?= W1, expand)),
	writeOK,

	write("f(a) ?= f(W)"),
	not(regle(f(a) ?= f(W2), rename)),
	writeOK
.

test_regle_orient() :-
	writeln("==== Regle : orient ===="),

	write("X ?= a"),
	not(regle(X1?=a, orient)),
	writeOK,

	write("X ?= f(a)"),
	not(regle(X2 ?= f(a), orient)),
	writeOK,

	write("X ?= Q"),
	not(regle(X3 ?=Q, orient)),
	writeOK,

	write("a ?= W"),
	regle(a ?= W1, orient),
	writeOK,

	write("f(a) ?= W"),
	regle(f(a) ?= W2, orient),
	writeOK,

	write("f(X) ?= W"),
	regle(f(X) ?= W4, orient),
	writeOK,

	write("f(a) ?= f(W)"),
	not(regle(f(a) ?= f(W3), orient)),
	writeOK
.
test_regle_decompose() :-
	writeln("==== Regle : decompose ===="),

	write("X ?= a"),
	not(regle( X ?= a, decompose)),
	writeOK,

	write("X ?= f(a)"),
	not(regle(X ?= f(a), decompose)),
	writeOK,

	write("X ?= Q"),
	not(regle(X?=Q, decompose)),
	writeOK,

	write("f(X) ?= W"),
	not(regle(f(X) ?= W, decompose)),
	writeOK,

	write("f(a) ?= g(a)"),
	not(regle(f(a) ?= g(a), decompose)),
	writeOK,

	write("f(a) ?= f(a,b)"),
	not(regle(f(a) ?= f(a, b), decompose)),
	writeOK,
	
	write("f(a,b) ?= f(a)"),
	not(regle(f(a,b) ?= f(a), decompose)),
	writeOK,
	
	write("f(a,g()) ?= f(b, c())"),
	regle(f(a, g()) ?= f(b, c()), decompose),
	writeOK,

	
	write("f(a, X) ?= f(a, b)"),
	regle(f(a, X) ?= f(a,b), decompose),
	writeOK,

	write("f(a) ?= f(W)"),
	regle(f(a) ?= f(W), decompose),
	writeOK,

	write("f(a, E, q) ?= f(W, c, e)"),
	regle(f(a) ?= f(W), decompose),
	writeOK
.

test_regle_clash() :-
	writeln("==== Regle : clash ===="),

	write("X ?= a"),
	not(regle(X?=a, clash)),
	writeOK,

	write("X ?= f(a)"),
	not(regle(X ?= f(a), clash)),
	writeOK,

	write("X ?= Q"),
	not(regle(X?=Q, clash)),
	writeOK,

	write("f(X) ?= W"),
	not(regle(f(X) ?= W, clash)),
	writeOK,

	write("f(a) ?= f(W)"),
	not(regle(f(a) ?= f(W), clash)),
	writeOK,

	write("f(Q, E) ?= g(E, V)"),
	regle(f(Q, E) ?= g(E, V), clash),
	writeOK,

	write("f(Q) ?= f(E, V)"),
	regle(f(Q, E) ?= g(E, V), clash),
	writeOK,

	write("f(x, y) ?= g(x, y)"),
	regle(f(x, y) ?= g(x, y), clash),
	writeOK,

	write("f(Q, E) ?= g(E, V, e)"),
	regle(f(Q, E) ?= g(E, V), clash),
	writeOK
.
test_regle_check() :-
	writeln("==== Regle : check ===="),

	write("X ?= a"),
	not(regle(X?=a, check)),
	writeOK,

	write("X ?= f(a)"),
	not(regle(X ?= f(a), check)),
	writeOK,

	write("X ?= Q"),
	not(regle(X?=Q, check)),
	writeOK,

	write("f(X) ?= W"),
	not(regle(f(X) ?= W, check)),
	writeOK,

	write("f(a) ?= f(W)"),
	not(regle(f(a) ?= f(W), check)),
	writeOK,

	write("f(Q, E) ?= g(E, V)"),
	not(regle(f(Q, E) ?= g(E, V), check)),
	writeOK,

	write("X ?= f(g(X))"),
	regle(X ?= f(g(X)), check),
	writeOK,

	write("X ?= f(X)"),
	regle(X ?= f(X), check),
	writeOK,

	write("X ?= f(a,X)"),
	regle(X ?= f(a,X), check),
	writeOK
.
\end{lstlisting}



\end{document}